الوحدة الأولى: المادة وتركيبها:

: درس المادة وخواصها
: درس تركيب المادة
: درس التركيب الذري للمادة

الوحدة الثانية: الطاقة:########

درس مصادر الطاقة وصورها.
درس تحولات الطاقة.
 درس الطاقة الحرارية.

الوحدة الثالثة: التنوع والتكيف في الكائنات الحية:########
درس تنوع الكائنات الحية ومبادئ تصنيفها.
 درس التكيف وتنوع الكائنات الحية.
____________________________________

توزيع منهج العلوم للمرحلة الإعدادية للترم الثاني

- الوحدة الأولي «التفاعلات الكيميائية» تتضمن:#######

- الدرس الأول الاتحاد الكيميائي.
- الدرس الثاني: المركبات الكيميائية.
- الدرس الثالث المعادلة الكيميائية والتفاعل الكيميائي.

########- الوحدة الثانية «القوي والحركة» تتضمن:
- الدرس الأول القوي الأساسية في الطبيعة.
- الدرس الثاني القوي المصاحبة للحركة.
- الدرس الثالث الحركة».

############# الوحدة الثالثة «الأرض والكون» تتضمن:
- الدرس الأول الأجرام السماوية.
- الدرس الثاني كوكب الأرض.
- الدرس الثالث الصخور والمعادن.
----------------------------------------------

4. حالات المادة
ثلاث حالات للمادة:
• الصلبة:
o لها حجم وشكل ثابتين.
o المسافات البينية صغيرة جدا.
o قوة التماسك الجزئية كبيرة جدا.
o حركة الجزيئات اهتزازية في مواضعها.
o أمثلة: الثلج، الحديد، الألمنيوم.
• السائلة:
o لها حجم ثابت وشكل غير ثابت.
o المسافة البينية كبيرة نسبيا
o قوة التماسك الجزئية ضعيفة.
o حركة الجزيئات كبيرة نسبيا.
o أمثلة: الماء، الكحول، الزيت.
• الغازية:
o ليس لها حجم أو شكل ثابتين.
o المسافات البينية كبيرة جدا.
o قوة التماسك الجزئية تكاد تكون منعدمة.
o حركة الجزيئات أكبر ما يمكن.
o أمثلة: بخار الماء، الأكسجين، ثاني أكسيد الكربون.